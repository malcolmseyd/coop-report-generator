% Discussion
\section{Discussion}

{{&ai.discussionIntro}}

{{#each ai.discussion}}
	\subsection{ {{&this.theme}} }
	{{&this.content}}
{{/each}}

% The main body of the report should contain a clear and informative description of what you accomplished on your work term. Include the approach, methodology, techniques or software used. Discuss any possible alternatives. Present any results and any data or information collected, possibly using tables, or figures. Any relevant background theory can be given.

% \begin{itemize}
%   \item The main body or text should be divided into numbered sections with headings. Emphasize the technical aspects. What problems did you encounter? What decisions did you make? What were the consequences of these decisions?
%   \item Do not include any non-technical personal experiences (such as social events, issues concerning transportation to and from the work place, or congeniality of co-workers).
% \end{itemize}

% \subsection{UVic Edge Logo}
% Figure \ref{fig:figure01} is an example of adding a figure. A figure should be referenced appropriately by number. The figure should also be discussed within the text and not be purely decorative. The UVic Edge Logo represents UVic \cite{uvicwebsite}.
% \vspace{0.75cm}

% \begin{figure}[h!]
%   \begin{center}
%     \includegraphics[scale=0.75]{figures/figure01.png}
%     \caption{UVic Edge Logo}
%     \label{fig:figure01}
%   \end{center}
% \end{figure}

% \subsubsection{Python Code}
% Figure \ref{fig:figure02} is an example of adding a code sample. The code block presents a function named function that prints ``Hello Work Term Report''.

% \begin{figure}[!ht]
%   \begin{lstlisting}[language=Python]
%                 def function():
%                  print(``Hello Work Term Report'')
%             \end{lstlisting}
%   \caption{Python Code}
%   \label{fig:figure02}
% \end{figure}

% \subsection{Example Table}
% Table \ref{table:table01} is an example of adding a table. The table showcases bold face text and centred elements. The table will appear in an appropriate location based on the remaining content.

% \begin{table}[ht]
%   \begin{center}
%     \begin{tabular}{|l|l|}
%       \hline
%       \textbf{Header 01} & \textbf{Header 02} \\ \hline
%       Text               & Descriptive Text   \\ \hline
%     \end{tabular}
%     \caption{Example Table}
%     \label{table:table01}
%   \end{center}
% \end{table}
